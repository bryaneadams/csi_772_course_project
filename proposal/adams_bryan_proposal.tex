\documentclass[12pt]{article}
\usepackage[top=0.8in, bottom=0.8in, left=0.8in, right=0.8in]{geometry}
\usepackage{amsmath}
\usepackage{amssymb}
\usepackage[T1]{fontenc}
\usepackage{hyperref}
\usepackage{graphicx}
\usepackage{makecell}
\usepackage{hyperref}

\pdfpagewidth 8.5in
\pdfpageheight 11.0in

\textheight = 700pt

\setlength{\parindent}{0pt}

\begin{document}

\section*{Abstract}

The proposed research is intended to develop a modeling framework to better understand how working groups within an organization, specifically Army Acquisitions, evolve over time. The research will focus on how a working group's skills, knowledge and abilities (SKAs) evolve over a given period. The modeling approach will require research into identifying what qualifies as a SKA, as well as, identifying and implementing methods to identify SKAs consistently within unstructured Army Acquisitions’ job postings, which is a set of free text documents. The research will provide the ability to create sets of required SKAs for a given acquisition team $j$ at period $t$, $\Omega_t^j$. Sets $\Omega_t^j$, will enable the creation of a modeling framework with the purpose of modeling an organization's working groups temporal evolution as it relates to SKAs.

\section{Relevant Literature}

Relevant literature to be reviewed can be broken into two distinct categories. First, identifying concepts and techniques for identifying and classifying SKAs. Identifying relevant SKAs has been extensively researched in economics, management and sociology. \cite{investment_human_capital, task_specific, on_the_mechanics, diversification}. Classifying SKAs is also extensively researched in many fields. \cite{specialization_career,industry-specific_human_capital, human_capital_specificity}. The second category of literature to be reviewed, but not yet identified, will pertain to Natural Language Processing (NLP) techniques to identify SKAs within free text and statistical methods for measuring similarity between SKAs.

\section{Data available}

Data for this research will come from two data sources. The first data source is 2.4 million historic job postings from \href{https://www.usajobs.gov}{USAJobs}. The historic job postings cover a period from 2017 - 2023. The subset to be analyzed are job postings related to Army Acquisitions, which accounts for approximately 555 thousand historic job postings over the given period. The second source of data to be used is employee data provided by a client organization. This data will be used to establish team transitions overtime; however, the data cannot be shared in its raw format.

\section{Proposed Methods}

First, an NLP model will be implemented to identify SKAs within historical job postings. Followed by methods to measure the similarity between SKAs and classify SKAs within a common SKA category. Finally, a modeling framework will be explored to better understand the temporal evolutions of the working groups as it pertains to SKAs.

\section{Potential Impact}

The impact from this research could take multiple forms. First, this research could provide organizations a method for forecasting SKA evolution within their working groups. Secondly, models explored will provide organizations the ability to predict emerging SKA requirements. Lastly, this research could provide organizations the ability to develop policies to effectively manage their workforce.

\newpage

\bibliographystyle{plain}
\bibliography{refs}

\end{document}

